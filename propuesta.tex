\documentclass{article}


\begin{document}

\begin{center}
{\bfseries\Large
  La ecuación de Poisson sobre geometrías complejas.
}

\vspace{0.25cm}

Integrantes: Andrés Díaz, Juan Pablo Flores, Sebastián Sánchez.

\vspace{0.5cm}

{\scriptsize
Proyecto para la versión 2024-2 del curso 
IMT3430 - Método para Ecuaciones Diferenciales, impartido por Manuel Sánchez.
}
\end{center}

Resolveremos la ecuación de Poisson en geometrías complejas, entendiendo
complejas como aquellas que tienen muchas puntas y agujeros.
En tales geometrías el mallado puede ser un cuello de botella a la hora
de aplicar elementos finitos, así también, su mala calidad puede
deteriorar la regularidad y fidelidad de la solución.
Para afrontar la situación descrita proponemos usar métodos 
aleatorizados~\cite{sawhney2020}
que no requieran de un mallado específico y que están basados en
formulaciones estocásticas de la ecuación diferencial.
Nuestro principal objetivo es comparar FEM con este método 
en geometrías complejas.

$$\text{probando}$$

\begin{thebibliography}{9}
\bibitem{sawhney2020}
Sawhney, Rohan, and Keenan Crane. "Monte Carlo geometry processing: A grid-free approach to PDE-based methods on volumetric domains." ACM Transactions on Graphics 39.4 (2020).
\end{thebibliography}

\end{document}
