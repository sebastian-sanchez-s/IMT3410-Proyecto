\documentclass{article}
\usepackage{amsmath, amssymb}
\usepackage[top=2cm]{geometry}

\begin{document}

\begin{center}
{\bfseries\Large
  La ecuación de Poisson sobre geometrías complicadas.
}

\vspace{0.25cm}

Integrantes: Andrés Díaz, Juan Pablo Flores, Sebastián Sánchez.

\vspace{0.5cm}

{\scriptsize
Proyecto para la versión 2024-2 del curso 
IMT3430 - Método para Ecuaciones Diferenciales, impartido por Manuel Sánchez.
}
\end{center}

En 1944 Shizuo Kakutani~\cite{kak44} demostró que la solución al problema de frontera
\begin{displaymath}
  -\Delta u = 0 \text{ en } \Omega, \quad
  u = g \text{ sobre } \partial\Omega
\end{displaymath}
se puede expresar puntualmente como el valor esperado \(\mathbb{E}(g(Y))\), donde $Y$ es el primer punto en $\partial \Omega$ alcanzado por un movimiento Browniano $W$ que comienza en el punto, $x_0$, donde queremos estimar la función.
Usando el método de caminata en esferas~\cite{sawhney2020} junto con estimaciones de 
Monte Carlo se puede estimar tal esperanza y por lo tanto la solución a la EDP.

La idea básica del método es que para estimar $u(x_0)$, nos basta con simular el paseo aleatorio del movimiento Browniano centrado en $x_0$ hasta llegar a $\partial \Omega$. Comenzamos la simulación envolviendo $x_0$ con una bola $B(x_0)\subseteq\Omega$. Luego, como el movimiento Browniano es simétrico, este puede escaparse por cualquier punto de la frontera de la bola de manera equiprobable. O sea, podemos estimar $u(x_0)$ con $u(X_1)$, donde $X_1\sim \mbox{Unif}(\partial B(x_0))$. Al no conocerse el valor de $u(X_1)$, este también debe ser estimado. Creamos así una secuencia aleatoria $X_1,\ldots,X_{N}$, la cual detenemos cuando nos acercamos lo suficiente al borde de $\Omega$; cuando $X_{N}\in \partial \Omega_\varepsilon$ (los puntos a distancia a lo más $\varepsilon >0$ de $\partial \Omega$), estimamos $u(X_N)$ mediante $g(\overline{X_N})$, donde $\overline{X_N}$ es el punto de $\partial \Omega$ más cercano a $X_N$. Escrito más sucintamente, la simulación es:

\begin{equation*}
    \hat{u}(X_k)=\begin{cases}
        g(\overline{X_k}), X_k\in\partial\Omega_{\varepsilon}\\
        \hat{u}(X_{k+1}),\,\text{en caso contrario.}
    \end{cases}
\end{equation*}

Debido a que la simulación extiende la condición de frontera a $\partial \Omega_{\varepsilon}$, esta se verá sesgada hacia aquellos puntos. Sin embargo, la magnitud de este sesgo estará controlada por el parámetro $\varepsilon>0$.

Es posible refinar el algoritmo usando técnicas estándar de reducción de varianza para disminuir el error inducido por la aleatoriedad de la simulación, siendo la manera más sencilla generar múltiples estimaciones de $u(x_0)$ y luego promediarlas.

La ventaja de esta formulación es que no requiere mallados complejos sobre el
dominio, a diferencia de FEM. Nuestra propuesta consiste en comparar ambos métodos
sobre geometrías complicadas, entendiendo complicadas como aquellas que tienen
muchas puntas y agujeros.

% Resolveremos la ecuación de Poisson en geometrías complejas, entendiendo
% complejas como aquellas que tienen muchas puntas y agujeros.
% En tales geometrías el mallado puede ser un cuello de botella a la hora
% de aplicar elementos finitos, así también, su mala calidad puede
% deteriorar la regularidad y fidelidad de la solución.
% Para afrontar la situación descrita proponemos usar métodos 
% aleatorizados~\cite{sawhney2020}
% que no requieran de un mallado específico y que están basados en
% formulaciones estocásticas de la ecuación diferencial.
% Nuestro principal objetivo es comparar FEM con este método 
% en geometrías complejas.

$$\text{probando}$$

\begin{thebibliography}{9}
\bibitem{kak44}
Kakutani, S. (1944). 143. two-dimensional brownian motion and harmonic functions. Proceedings of the Imperial Academy, 20(10), 706-714.
\bibitem{sawhney2020}
Sawhney, Rohan, and Keenan Crane. "Monte Carlo geometry processing: A grid-free approach to PDE-based methods on volumetric domains." ACM Transactions on Graphics 39.4 (2020).
\end{thebibliography}

\end{document}
