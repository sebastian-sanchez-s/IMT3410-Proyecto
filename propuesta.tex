\documentclass{article}

\usepackage{amsmath, amssymb}
\usepackage[top=2cm]{geometry}

\begin{document}

\begin{center}
{\bfseries\Large
  La ecuación de Poisson sobre geometrías complejas.
}

\vspace{0.25cm}

Integrantes: Andrés Díaz, Juan Pablo Flores, Sebastián Sánchez.

\vspace{0.5cm}

{\scriptsize
Proyecto para la versión 2024-2 del curso 
IMT3430 - Método para Ecuaciones Diferenciales, impartido por Manuel Sánchez.
}
\end{center}

En 1944 Shizuo Kakutani~\cite{kak44} demostró que la solución al problema de frontera
\begin{displaymath}
  -\Delta u = 0 \text{ en } \Omega, \quad
  u = g \text{ sobre } \partial\Omega
\end{displaymath}
se puede expresar puntualmente como el valor esperado \(\mathbb{E}(g(W))\)
de la primera vez un movimiento Browniano \(W\) toca el borde comenzando en el punto.
Usando el método de caminata en esferas~\cite{sawhney2020} junto con estimaciones de 
Montercarlo se puede simular tal esperanza y por lo tanto la solución a la EDP.
La ventaja de esta formulación es que no requiere mallados complejos sobre el
dominio, a diferencia de FEM. Nuestra propuesta consiste en comparar ambos métodos
sobre geometrías complicadas, entendiendo complicadas como aquellas que tienen
muchas puntas y agujeros.

% Resolveremos la ecuación de Poisson en geometrías complejas, entendiendo
% complejas como aquellas que tienen muchas puntas y agujeros.
% En tales geometrías el mallado puede ser un cuello de botella a la hora
% de aplicar elementos finitos, así también, su mala calidad puede
% deteriorar la regularidad y fidelidad de la solución.
% Para afrontar la situación descrita proponemos usar métodos 
% aleatorizados~\cite{sawhney2020}
% que no requieran de un mallado específico y que están basados en
% formulaciones estocásticas de la ecuación diferencial.
% Nuestro principal objetivo es comparar FEM con este método 
% en geometrías complejas.

\begin{thebibliography}{9}
\bibitem{kak44}
Kakutani, S. (1944). 143. two-dimensional brownian motion and harmonic functions. Proceedings of the Imperial Academy, 20(10), 706-714.
\bibitem{sawhney2020}
Sawhney, Rohan, and Keenan Crane. "Monte Carlo geometry processing: A grid-free approach to PDE-based methods on volumetric domains." ACM Transactions on Graphics 39.4 (2020).
\end{thebibliography}

\end{document}
